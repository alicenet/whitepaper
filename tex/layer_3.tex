\section{Layer Three}

\subsection{Scriptless Scripts and State Channels}

An important implication of this work is one that was not fully
considered at the time of conception.
The realization came later when thinking about how the accusation and
snapshot formation process operates.
Namely, the inclusion of the abstraction of signature verification may
be extended in further iterations to allow Schnorr signatures and other
primitives.
With this toolbox in place and the ability to extend the validation of
a UTXO in an isolated manner, as is afforded by the design of the
account abstraction system, more exotic primitives may be easily built.
These include adaptor signatures, SE-snarks, and many other primitives
that may be leveraged to accommodate the concept of scriptless scripts.
Ultimately these constructs allow the system we have defined to be used
as a restricted smart contract capable system without the need to
verify the actual smart contract logic in the chain.

Further, other protocols may leverage these capabilities in concert
with the existing ability to compactly prove state about the sidechain
in the context of an Ethereum smart contract.
The authors of this paper believe this ultimately may serve as an
interesting primitive for constructing proofs of verifiable computation
within an Ethereum smart contract.
An example of this process that may be built based on the operation of
the system without any modification follows.

State channels are a promising technology that suffer from a problem of
requiring all participants to be online at all times.
This problem may be addressed through a hybrid solution based on the
construction of \LayerTwoLong{}.
Given the existence of \LayerTwoLong{}, a state channel may be constructed
using a negotiated BLS multisignature account as described in the
account abstraction section.

Let these two parties be named Alice and Bob.
Let Alice and Bob have some motivation for continual exchange of value.
Alice and Bob coordinated to negotiate a 2 out of 2 BLS multisignature.

Alice will be the first mover in this example, and so Alice will create
a smart contract on Ethereum in which she writes the negotiated public
key and deposits some value.
Let this contract have knowledge of the snapshot contract for
\LayerTwoLong{}.
Let this smart contract have three exit conditions.
The first exit should be an exit condition that allows Alice to remove
the deposit if Bob does not also place an equal deposit as Alice by
some predetermined block number in the future.
Let the other two exit conditions only be accessible after both parties
have placed such a deposit.
Let these other two exit conditions be an exit initiated by Alice or an
exit initiated by Bob.
Let an exit by Alice or Bob only occur within ten Ethereum blocks of a
snapshot being recorded to the Ethereum blockchain by the validators of
\LayerTwoLong{}.
Let this exit require a proof of state against the snapshot and only be
valid for the most recent snapshot of \LayerTwoLong{}.

Before Bob places a deposit, let Alice and Bob coordinate to create a
\DataStore{} in \LayerTwoLong{} that holds a single 32 byte value.
Let this value be the balance of Alice’s deposit in the Ethereum
smart contract.
Once this datastore has been constructed, let Bob place his deposit
into the smart contract.
Each time Alice and Bob wish to exchange value let Alice and Bob
coordinate to sign a new transaction that overwrites the previous
\DataStore{} with a new value.
This operation should not occur in any epoch until at least fifteen
blocks have been recorded in the Ethereum blockchain since the last
snapshot.
This delay is to ensure the window of generating a valid exit proof for
the smart contract is fully terminated before a new round of exchange
may begin.

Let the value of the \DataStore{} be the new balance of Alice after a
mutually agreed upon exchange of value.
What governs such exchange is beyond the scope of this example.
Let us simply say that they agree to exchange value based on
information external to the system.
For each state transition, both parties must sign a partial signature
to a transaction that is written to the sidechain.
Thus, the value is mutually agreed upon.

If at any point either party wishes to terminate the channel, the party
may stop signing additional changes to the \DataStore{}.
At the termination of an epoch, the data of this \DataStore{} may be
proven in the Ethereum Blockchain in a compact manner using the
parameters of the most recent snapshot.
The proof of existence for the \DataStore{} UTXO is just a Merkle Proof of
Inclusion against the StateRoot of the snapshot block.
The parsing logic of the \DataStore{} is fairly trivial assuming the
RawData field of the \DataStore{} is a fixed size object.
This assumption will allow fixed offset parsing of the canonical
encoding of the \DataStore{} object.

Once the existence of a UTXO is proven against a snapshot, the public
key of the signer must be proven as equal to the 2 out of 2 BLS public
key as stored in the smart contract.
Assuming economic security of the sidechain is sufficient, the BLS
signature of the \DataStore{} need not be verified in the smart contract
since this requirement is enforced in the sidechain validation logic.
If this assumption is insufficient, our system is capable of validating
a BLS signature in an Ethereum smart contract for less than 200K gas.
This can likely be bounded more closely to 150K gas, but 200K is a safe
upper bound for what we are describing.

Alice or Bob may initiate such a termination in any epoch in the
designated window.
Thus, the final state of the external system may be collapsed and
proven in the context of the EVM.
This state is the final state of the system which is, by definition,
the last state that both parties have agreed to in advance.
Given that this state is the balance of Alice, the smart contract may
enforce that Alice may withdraw only this balance at the termination of
the channel.
Bob may only withdraw his own value minus the delta of Alice’s initial
value.

The operation of parsing the object, validating the Merkle proof, and
checking the data field as a simple uint256 value should cost around
700K gas for a Merkle Proof of 256 keys.
This is the most complex Merkle Proof possible and actual costs are
likely to be a fraction of this worst case value.
Specifically, the cost should scale as $O(\log n)$ of the number of UTXOs
that exist in the State Trie.

This construct is a toy example, but it does highlight a core benefit
of the protocol.
Namely, that the problems around state channel participant availability
may be addressed by preventing an exit from a state both parties have
not agreed to as the most recent state of the system.
Other constructs may be devised for more complex cases.
The fundamental idea is that \LayerTwoLong{} may act as an out-of-band
storage system that is cryptographically verifiable and allows proof of
an item being the most recent object state.

\subsection{AdLedger PKI}

One application of \LayerTwoLong{} is Transparent by Design Public Key
Infrastructure (TPKI).
This is discussed more fully in its own whitepaper, but we summarize
the main ideas here.

The need for TPKI comes from the unfortunate but undeniable truth:
the x.509 standard is broken beyond repair.
This comes from ill-defined standards as well as haphazard attempts
at patchwork extensions to salvage it.
One main issue arises from certificate revocation, as it is not
possible to ensure that invalid certificates are rejected.
A proposed solution is OCSP, but it is not sufficient due to its
inability to handle the large number of requests.

To counteract this, we developed TPKI.
Instead of issuing certificates with the lifetime of years that
may become compromised, we issue short-lived certificates (validity
around one day) that are regularly renewed.
Having short-lived certificates ensures auto-revocation and requires
proof of continued compliance for reissuance.
Additionally, each certificate has well-defined properties and we
do not allow arbitrarily chaining of certificates; this protects
us against some known attacks.
Furthermore, any certificate not present or ill-formed is considered
invalid by default and should not be trusted.
Root level certificates are not cross-validated (although they
are self-signed) to ensure a well-defined notion of validity
and revocation.
Should a root certificate become compromised, a predetermined storage
location in \LayerTwoLong{} will hold the revocation key.
This revocation public key, if present, will show that all users
should reject subordinate certificates from this root certificate.
We envision there being multiple Certificate Authorities with separate
root certificates so that the revocation of one will not affect
the others.

