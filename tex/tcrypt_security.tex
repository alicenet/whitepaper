\subsection{Cryptographic Security}
\label{ssec:security}

\chgcomment{Not really sure if we need a specific section on this
but I will keep it for now.}

Security is essential, so we will discuss those results here.

Throughout the procedure, if there is ever cryptographic proof of
malicious action of $P_{i}\in\mathcal{P}$, then $P_{i}$'s stake
will be slashed and the entire process will restart.
This is different from the process described in~\cite{ethdkg};
there, bad actors of $\mathcal{P}$ are excluded from $\mathcal{Q}$
and any bad participant in $\mathcal{Q}$ who refuses to share
the necessary information can have it reconstructed by the other
participants.
We take a stronger stance because we want the threshold at the
beginning [$(t,n)$ where $t = \ceil{2n/3}-1$] to stay the same
throughout the entire process; that is, we require
$\mathcal{P} = \mathcal{Q}$.
As noted above, this is not strictly necessary in the second stage
because we have all of the information necessary to reconstruct
the secret $s_{i}$, but we take a firm stand against malicious
behavior.

