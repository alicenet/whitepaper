\subsection{Distributed Key Generation Protocol}
\label{ssec:dkg}

We desire a Byzantine Fault Tolerant consensus algorithm.
So, we let $\mathcal{P}$ be the total collection of participants with
$\abs{\mathcal{P}} = n$.
We set the threshold $t = \floor{2n/3}$ in our $\parens{t,n}$
secret sharing protocol.
Thus, it takes $t+1$ users to reconstruct a secret, which corresponds
to strictly greater than two-thirds of the participants.
We assume there is an open broadcast channel between all participants.
Encryption will be provided through Diffie-Hellman-style shared
secret encryption; this will be discussed at the end of
Sec.~\ref{sssec:share_distribution}.
The group's shared secret, henceforth called the \emph{master secret key},
will be the sum of the shared secrets of
each group member who correctly shared his secret.
Once there are $t+1$ valid partial signatures, these will be
combined to form a group signature.

As stated above, although the final master public key will
reside in $\G_{2}$, because the precompiled contracts
currently available in the Ethereum Virtual Machine only allow addition
and scalar multiplication in $\G_{1}$ (multiplication and exponentiation
in our multiplicative notation), we will primarily
use computations in $\G_{1}$ and anything required
in $\G_{2}$ will be confirmed via a \textsc{PairingCheck} call.

\subsubsection*{Protocol Overview}

We briefly give an overview of the DKG protocol.
Here are the different phases of the protocol:

\begin{enumerate}
\item \hyperref[sssec:registration]{\textbf{\Registration{}:}}
    Participants submit a public key
    which will allow for secure communication on an open channel.
\item \hyperref[sssec:share_distribution]{\textbf{\ShareDistribution{}:}}
    Secret shares are distributed
    through verifiable secret sharing;
    these shares allow for the construction of the master public key.
\item \hyperref[sssec:share_distribution_dispute]{\textbf{\ShareDistributionDispute{}:}}
    If there are any invalid shares,
    accusations can be made against malicious validators.
\item \hyperref[sssec:key_share_submission]{\textbf{\KeyShare{}:}}
    Portions of the master public key and accompanying
    proofs are submitted.
\item \hyperref[sssec:mpk_submission]{\textbf{\MasterPublicKey{}:}}
    Any participant is able to construct and submit
    the master public key.
\item \hyperref[sssec:gpkj_submission]{\textbf{\GPKSubmission{}:}}
    Group public keys are submitted.
\item \hyperref[sssec:gpkj_dispute]{\textbf{\GPKDispute{}:}}
    Accusations are made against those who submitted invalid
    group public keys.
\item \hyperref[sssec:gpkj_complete]{\textbf{\Completion{}:}}
    Any participant is able to signal the successful completion
    of the DKG protocol.
    This is required for MadNet to proceed.
\end{enumerate}



\subsubsection{\Registration{}}
\label{sssec:registration}
Each participant $P_{i}\in\mathcal{P}$ begins by selecting
a secret key $\sk_{i}\in\Z_{q}$ with public key $\pk_{i} = g_{1}^{\sk_{i}}$.
The public-private key pair $\angles{\pk_{i},\sk_{i}}$ will be
used for secure communication over the insecure broadcast channel;
it will not be used for signing any messages.
The public key $\pk_{i}$ will be submitted during registration.



\subsubsection{\ShareDistribution{}}
\label{sssec:share_distribution}
The primary purpose of this phase is to correctly share secrets
between validators.
These shared secret will then be used to establish a master secret key.
By distributing the appropriate shares,
each participant will be able to construct a portion
of the key to create group signatures.
This is done via Verifiable Secret Sharing
and is based on a distributed version of
Shamir's Secret Sharing~\cite{shamir1979share}.

Participant $P_{i}$ chooses a secret
$s_{i}\in\Z_{q}$ to share with the other participants;
the secret $s_{i}$ should be a
\emph{cryptographically-secure} pseudorandom number.
To do this, choose a secret polynomial $f_{i}:\Z_{q}\to\Z_{q}$
with

\begin{equation}
    f_{i}(x) = c_{i0} + c_{i1}x + c_{i2}x^{2} + \cdots + c_{it}x^{t},
\end{equation}

\noindent
where $c_{i0} = s_{i}, c_{i1}, \cdots, c_{it}$ are chosen uniformly
in $\Z_{q}$.
The coefficients $\braces{c_{ij}}_{j}$ should also be 
\emph{cryptographically-secure} pseudorandom numbers.
Setting

\begin{equation}
    C_{ik} = g_{1}^{c_{ik}}\quad k\in\braces{0,\cdots,t},
\end{equation}

\noindent
we have the corresponding public polynomial $F_{i}:\Z_{q}\to\G_{1}$:

\begin{equation}
    F_{i}(x) = C_{i0}C_{i1}^{x}\cdots C_{it}^{x^{t}}.
\end{equation}

\noindent
The shared secret from $P_{i}$ to $P_{j}$ is

\begin{equation}
    s_{i\to j} = f_{i}(j)
\end{equation}

\noindent
and

\begin{equation}
    \overline{\texttt{s}}_{i\to j} =
        \textsc{Encrypt}(\text{sk}_{i},\text{pk}_{j},j, s_{i\to j})
\end{equation}

\noindent
is the encrypted secret.
Here, \textsc{Encrypt} refers to a particular encryption scheme discussed
at the end of this \hyperref[par:secret_enc]{section}.
Participant $P_{i}$ will broadcast the message

\begin{equation}
    \left\{ 
        \overline{\texttt{s}}_{i\to 1}, \overline{\texttt{s}}_{i\to 2},
            \cdots,
            \overline{\texttt{s}}_{i\to i-1},
            \overline{\texttt{s}}_{i\to i+1},
            \cdots,
            \overline{\texttt{s}}_{i\to n},
        C_{i0}, C_{i1}, \cdots, C_{it}
    \right\}
\end{equation}

\noindent
over the broadcast channel.
In our setting, this information will be submitted to the
Ethereum smart contract and
all participants will receive notification.
We note this message does not include the secret
$\overline{\texttt{s}}_{i\to i}$.

Because it would be costly to store the entire submitted information
on Ethereum, we instead store the corresponding hash value.
To be precise: we first hash all the encrypted shares
and the commitments separately before concatenating the result
and hashing once more; this amounts to a small ``Merkle tree''.
More explicitly, we compute the following:

\begin{align}
    \texttt{encryptedHash} &= \Keccak{}\parens{
        \overline{\texttt{s}}_{i\to 1}, \overline{\texttt{s}}_{i\to 2},
            \cdots,
            \overline{\texttt{s}}_{i\to i-1},
            \overline{\texttt{s}}_{i\to i+1},
            \cdots,
            \overline{\texttt{s}}_{i\to n}} \nonumber\\
        \texttt{commitmentsHash} &= \Keccak{}\parens{C_{i0}, C_{i1}, \cdots,
            C_{it}}
        \nonumber\\
    \texttt{distributedHash} &= \Keccak{}\parens{\texttt{encryptedHash},
                                    \texttt{commitmentsHash}}.
\end{align}

\noindent
Here, \Keccak{} is the standard hash function on Ethereum.
The hash value \texttt{distributed\-Hash} is then stored in the
smart contract with each participant.
Whenever accusation logic needs to be performed,
the broadcast information may be resubmitted and its hash value
recomputed before comparing it to the stored hash value.
This ensures that malicious parties are unable to submit invalid
information to falsely accuse honest participants.

Once participant $P_{j}$ receives the message from $P_{i}$,
he sets

\begin{equation}
    \hat{s}_{i\to j} = \textsc{Decrypt}(\text{sk}_{j},\text{pk}_{i}, j,
        \overline{\texttt{s}}_{i\to j}).
\end{equation}

\noindent
$P_{j}$ then determines if

\begin{equation}
    g_{1}^{\hat{s}_{i\to j}} \overset{?}{=} F_{i}(j).
    \label{eq:secret_share_test}
\end{equation}

\noindent
If we have equality, then $\hat{s}_{i\to j} = s_{i\to j}$.
Otherwise, $P_{i}$ incorrectly shared his secret.

\paragraph{Shared Secret Encryption}
\label{par:secret_enc}
As mentioned above, the shared secret from
$P_{i}$ to $P_{j}$ is $s_{i\to j} = f_{i}(j)$.
To encrypt this, we compute a shared secret key
by the Diffie-Hellman key exchange~\cite[Chap.~12.6.1]{hac1996}:

\begin{equation}
    k_{ij} = \pk_{i}^{\sk_{j}} = \pk_{j}^{\sk_{i}} =
    g_{1}^{\sk_{i}\sk_{j}}.
\end{equation}

\noindent
Encryption and decryption are based on the idea of a stream cipher;
in particular, we cryptographically hash
the $x$-coordinate of the shared secret key along with the index
of the participant receiving the message as our stream.
By including the index of the intended recipient in the hash function,
each stream is unique.
See Alg.~\ref{alg:enc_dec} for details.

\begin{algorithm}[t]
\caption{Encryption and decryption functions}
\label{alg:enc_dec}
\begin{algorithmic}[1]
\Function{Encrypt}{$\sk$,$\pk$,$j$,$s$}
    \Comment{Encrypt secret $s$ to participant $j$;
        $\pk$ is $j$'s public key}
    \State $k = \pk^{\sk}$
    \Comment{k is the shared secret}
    \State $\texttt{kX}  = \texttt{u2b}(k_{x})$
    \Comment{Convert $x$ coordinate of shared secret to bytes}
    \State $\texttt{j}   = \texttt{u2b}(j)$
    \State $\texttt{s}   = \texttt{u2b}(s)$
    \State $\texttt{HKj} = H(\texttt{kX} || \texttt{j})$
    \State $\overline{\texttt{s}} = \texttt{s} \oplus \texttt{HKj}$
    \State \Return $\overline{\texttt{s}}$
\EndFunction

\State
\Function{Decrypt}{$\sk$,$\pk$,$j$,$\overline{\texttt{s}}$}
    \Comment{Decrypt secret $\overline{\texttt{s}}$ to participant $j$;
        $\sk$ is $j$'s secret key}
    \State $k = \pk^{\sk}$
    \Comment{k is the shared secret}
    \State $\texttt{kX}  = \texttt{u2b}(k_{x})$
    \Comment{Convert $x$ coordinate of shared secret to bytes}
    \State $\texttt{j}   = \texttt{u2b}(j)$
    \State $\texttt{HKj} = H(\texttt{kX} || \texttt{j})$
    \State $\texttt{s}   = \overline{\texttt{s}} \oplus \texttt{HKj}$
    \State $s            = \texttt{b2u}(\texttt{s})$
    \State \Return $s$
\EndFunction

\State
\Function{DecryptSS}{$k$,$j$,$\overline{\texttt{s}}$}
    \Comment{Decrypt secret $\overline{\texttt{s}}$ to participant $j$}
    \State $\texttt{kX}  = \texttt{u2b}(k_{x})$
    \Comment{Convert $x$ coordinate of shared secret to bytes}
    \State $\texttt{j}   = \texttt{u2b}(j)$
    \State $\texttt{HKj} = H(\texttt{kX} || \texttt{j})$
    \State $\texttt{s}   = \overline{\texttt{s}} \oplus \texttt{HKj}$
    \State $s            = \texttt{b2u}(\texttt{s})$
    \State \Return $s$
\EndFunction
\end{algorithmic}
\end{algorithm}




\subsubsection{\ShareDistributionDispute{}}
\label{sssec:share_distribution_dispute}
If we have $g_{1}^{\hat{s}_{i\to j}} = F_{i}(j)$
in Eq.~\eqref{eq:secret_share_test} for all participants,
then everyone correctly shared his secrets.
In this case, there is nothing to dispute and the DKG protocol
will proceed to the next phase.

Instead, we now suppose that $\overline{\texttt{s}}_{i\to j}$ is incorrect;
that is, $g_{1}^{\hat{s}_{i\to j}} \ne F_{i}(j)$
in Eq.~\eqref{eq:secret_share_test} for some $i$ and $j$.
In order to prove this to be the case, everyone needs to be
able to prove that the encrypted secret
$\overline{\texttt{s}}_{i\to j}$ is incorrect.
To do this, $P_{j}$ must publish and prove the shared secret $k_{ij}$;
this is required in order to ensure bad actors
do not submit false proofs against honest actors.

Proving $k_{ij}$ is the shared secret is based on showing

\begin{equation}
    \pk_{j} = g_{1}^{\sk_{j}} \quad\text{and}\quad
    k_{ij} = \pk_{i}^{\sk_{j}}
\end{equation}

\noindent
\emph{without} sharing the secret key $\sk_{j}$;
that is, we wish to show
$\dlog_{g_{1}}(\pk_{j}) = \dlog_{\pk_{i}}(k_{ij})$
while keeping their common value ($P_{j}$'s secret key $\sk_{j}$) secret.
To do this, we use a zero-knowledge proof;
see Alg.~\ref{alg:zk_dleq_proof} for constructing the zk-proof
and Alg.~\ref{alg:zk_dleq_verify} for proof verification.
One reference for zk-proofs involving discrete logarithms
is~\cite{camenisch1997proof}.

Thus, $P_{j}$ would compute

\begin{equation}
    \pi(k_{ij}') = \textsc{DLEQ}(g_{1},\pk_{j},\pk_{i},k_{ij}',\sk_{j})
\end{equation}

\noindent
and publish $\angles{k_{ij}',\pi(k_{ij}')}$, where $k_{ij}'$
is claimed shared secret.
This allows anyone to use \textsc{DLEQ-verify} to determine
its validity.
If

\begin{equation}
    \textsc{DLEQ-verify}(g_{1},\pk_{j},\pk_{i},k_{ij}',\pi(k_{ij}'))
        = \texttt{true},
\end{equation}

\noindent
then $k_{ij}' = k_{ij}$, the shared secret.
Using this, everyone can decrypt $\overline{\texttt{s}}_{i\to j}$
by

\begin{equation}
    \hat{s}_{i\to j}
        = \textsc{DecryptSS}(k_{ij},j,\overline{\texttt{s}}_{i\to j}, b)
\end{equation}

\noindent
and determine if

\begin{equation}
    g_{1}^{\hat{s}_{i\to j}} \overset{?}{=} F_{i}(j).
    \label{eq:secret_share_test_2}
\end{equation}

\noindent
If the DLEQ proof $\pi(k_{ij}')$ shows $k_{ij}'$ is the shared
secret between $P_{j}$ and $P_{i}$ and we do not have equality in
Eq.~\eqref{eq:secret_share_test_2}, then $P_{i}$ is acted maliciously
and should be removed.
There are two other possibilities:
$k_{ij}'$ is not the shared secret, or $k_{ij}'$ is the shared secret
and we have equality in Eq.~\eqref{eq:secret_share_test_2}.
In both cases, $P_{j}$ acted maliciously and should be removed.
Thus, when $P_{j}$ submits a claim that $P_{i}$
failed to share a secret,
either $P_{j}$'s or $P_{i}$'s stake will be slashed.

In practice, $P_{j}$ will submit $P_{i}$'s broadcast message to an
Ethereum smart contract along with purported shared secret $k_{ij}'$
and proof $\pi(k_{ij}')$, and the smart contract would
determine its validity and burn stake as appropriate.

\input{algs/zk_dleq.tex}



\subsubsection{\KeyShare{}}
\label{sssec:key_share_submission}
At this point, we know that every individual correctly shared his secret.
Thus, we can now use this to submit the information required
to build the master public key.

Participant $P_{i}$'s secret $s_{i}$ is concealed behind
their commitment $C_{i0} = g_{1}^{s_{i}}$.
As discussed in~\cite{gennaro3revisiting,gennaro1999secure,ethdkg},
in order to ensure that no bad actors gain any information
about the master public key and are not be able to change
its underlying probability distribution, we require
$h_{1}\in\G_{1}$ such that $\dlog_{g_{1}}h_{1}$ is unknown.
We also let $h_{2}\in\G_{2}$ be a generator.

Let

\begin{equation}
    \pi(h_{1}^{s_{i}}) = \textsc{DLEQ}(
        g_{1},g_{1}^{s_{i}},h_{1},h_{1}^{s_{i}},s_{i})
\end{equation}

\noindent
be the zk-proof of discrete logarithm equality.
Because $C_{i0} = g_{1}^{s_{i}}$ is public knowledge
(stored in the smart contract)
and $P_{i}$ correctly shared his secret $s_{i}$,
the zk-proof will ensure the submitted value is valid.
Additionally, $P_{i}$ will publish $h_{2}^{s_{i}}$
(his portion of the master public key)
so that we can ensure

\begin{equation}
    \textsc{PairingCheck}(h_{1}^{s_{i}},\bar{h}_{2},h_{1},h_{2}^{s_{i}})
        = \texttt{true}.
\end{equation}

\noindent
This \textsc{PairingCheck} will be called by a smart contract.
Thus, failure of $P_{i}$ to publish $h_{1}^{s_{i}}$,
a valid proof $\pi(h_{1}^{s_{i}})$, and the corresponding
$h_{2}^{s_{i}}$ amounts to misbehavior which will
lead to a fine.
In particular, $P_{i}$ will submit
$\angles{h_{1}^{s_{i}}, \pi(h_{1}^{s_{i}}), h_{2}^{s_{i}}}$
to the smart contract which will verify the zk-proof.
The Ethereum smart contract will store $h_{1}^{s_{i}}$ from
all participants and broadcast
$h_{1}^{s_{i}}$, $\pi(h_{1}^{s_{i}})$, and $h_{2}^{s_{i}}$.

If any participant does not correctly submit these values,
then this entire process is required.



\subsubsection{\MasterPublicKey{}}
\label{sssec:mpk_submission}
In this phase, a participant must submit the master public key.

At this phase, all participants have correctly shared
the secret share $s_{i}$ as well as $h_{1}^{s_{i}}$ and $h_{2}^{s_{i}}$.
The individual shared secrets $s_{i}$ allow us to define the
\emph{master secret key} $\text{msk}$:

\begin{equation}
    \text{msk} = \sum_{P_{i}\in\mathcal{P}} s_{i}.
\end{equation}

\noindent
This gives us the \emph{master public key} $\text{mpk}$:

\begin{align}
    \text{mpk} &= h_{2}^{\text{msk}} \nonumber\\
        &= \prod_{P_{i}\in\mathcal{P}} h_{2}^{s_{i}}.
\end{align}

\noindent
A Byzantine-fault tolerant subgroup $\mathcal{R}\subseteq\mathcal{P}$ can
correctly obtain the secret $s_{i}$ via Lagrange interpolation:

\begin{align}
    s_{i} &= \sum_{P_{j}\in\mathcal{R}} s_{i\to j} R_{j} \nonumber\\
    R_{j} &= \prod_{\substack{P_{k}\in\mathcal{R} \\ k\ne j}} \frac{k}{k-j}.
    \label{eq:Rj_coefs}
\end{align}

\noindent
This would allow us to recover the secret $s_{i}$ should $P_{i}$
fail to share $h_{1}^{s_{i}}$ below; however,
we take a stricter response and would view failure to share
as malicious activity leading to stake slashing.
Furthermore, the DKG process must restart.
We note that the Lagrange interpolation just described
occurs in a finite field;
this means that $R_{j}$ is constructed using finite field divisions.

Because $h_{2}^{s_{i}}$ has been broadcast to all participants,
anyone will be able to submit

\begin{equation}
    \text{mpk} = \prod_{P_{i}\in\mathcal{P}} h_{2}^{s_{i}}
\end{equation}

\noindent
to the smart contract.
Because $\braces{h_{1}^{s_{i}}}_{i\in\mathcal{P}}$ are stored
and valid, the contract can construct

\begin{equation}
    \text{mpk}^{*} = \prod_{P_{i}\in\mathcal{P}} h_{1}^{s_{i}}
    \label{eq:mpk_dual}
\end{equation}

\noindent
and confirm its validity with a \textsc{PairingCheck}
with the follow operation:

\begin{equation}
    \textsc{PairingCheck}(\text{mpk}^{*},\bar{h}_{2},h_{1},\text{mpk})
        = \texttt{true}.
\end{equation}

\noindent
The master public key can then be stored publicly and used
for group signature verification.



\subsubsection{\GPKSubmission{}}
\label{sssec:gpkj_submission}
At this point, we have successfully constructed the master public
key $\text{mpk}$ and distributed
the master secret key $\text{msk}$ among the members of $\mathcal{P}$.
We now turn our attention to constructing group
signatures from partial signatures.

Each participant $P_{j}\in\mathcal{P}$ has a portion of the
master secret key; this portion is called the \emph{group secret key}
$\gsk_{j}$:

\begin{equation}
    \gsk_{j} = \sum_{P_{i}\in\mathcal{P}} s_{i\to j}.
\end{equation}

\noindent
This is possible because we proved that every participant
in $\mathcal{P}$ correctly shared his secret share.
We note that that $s_{j\to j}$ is included in the sum for
$\gsk_{j}$ even though the encrypted form was not publicly shared.
Naturally, there is the corresponding \emph{group public key}
$\gpk_{j}$:

\begin{equation}
    \gpk_{j} = h_{2}^{\gsk_{j}}.
\end{equation}

\noindent
Here, $\gpk_{j}$ is $P_{j}$'s portion of the master public key
and will be broadcast to all users.
Cryptographic proof that $\gpk_{j}$ is valid will be discussed
in Sec.~\ref{sssec:gpkj_dispute}.



\subsubsection{\GPKDispute{}}
\label{sssec:gpkj_dispute}
We now look at how to ensure the broadcast value of
$\gpk_{j}$ is valid.

Along with the $P_{j}$'s group public key $\gpk_{j}\in\G_{2}$,
there is the corresponding dual version in $\G_{1}$:

\begin{align}
    \gpk_{j}^{*} &= g_{1}^{\gsk_{j}}
            \nonumber\\
        &= \prod_{P_{i}\in\mathcal{P}}F_{i}(j).
    \label{eq:gpkj_star_def}
\end{align}

\noindent
Note the base is $g_{1}$ and not $h_{1}$ as in the case of $\text{mpk}^{*}$
in Eq.~\eqref{eq:mpk_dual}.
Here, we remember that

\begin{equation}
    F_{i}(j) = C_{i0}C_{i1}^{j}C_{i2}^{j^{2}}\cdots C_{it}^{j^{t}}.
    \label{eq:Fij_def}
\end{equation}

\noindent
This shows that $\gpk_{j}^{*}$ is public knowledge
and can be used to determine validity;
that is, we form $\gpk_{j}^{*}$ locally and the verify that

\begin{equation}
    \textsc{PairingCheck}(\gpk_{j}^{*}, \bar{h}_{2}, g_{1}, \gpk_{j})
        = \texttt{true}.
\end{equation}

\noindent
If all $\gpk_{j}$ submissions are valid,
then there is nothing to dispute and we can proceed to the next phase.
Otherwise, some accusations must be made to prove malicious behavior.

\paragraph{Gas Cost Discussion}
The major costs will be calling the precompiled contracts
\textsc{ECAdd}, \textsc{ECMul}, and \textsc{PairingCheck};
see Table~\ref{tab:evm_gas_cost} for the specific gas costs.
Note that the cost for \textsc{PairingCheck} comes
from our assumption that we are testing 2 pairings.
These costs come from EIP-1108\footnote{
    \url{https://github.com/ethereum/EIPs/blob/master/EIPS/eip-1108.md}}.

\begin{table}
\centering
\begin{tabular}{|c|c|}
\hline
Precompile & Gas Cost \\
\hline
\hline
\textsc{ECAdd} & 150 \\
\textsc{ECMul} & 6000 \\
\textsc{PairingCheck} &  113000 \\
\hline
\end{tabular}
\caption[EVM Gas Cost]{Gas cost of important precompiled contracts on the
    Ethereum Virtual Machine.}
\label{tab:evm_gas_cost}
\end{table}




\paragraph{GPK Accusation}
To perform the accusation, we submit the necessary information
to compute $\gpk_{j}^{*}$ to a smart contract and then show
that the submission fails the pairing check.

Using Eqs.~\eqref{eq:gpkj_star_def} and \eqref{eq:Fij_def},
we see

\begin{align}
    \gpk_{j}^{*} &= \prod_{P_{i}\in\mathcal{P}}F_{i}(j)
            \nonumber\\
        &= \prod_{P_{i}\in\mathcal{P}} \brackets{\prod_{k=0}^{t} C_{ik}^{j^{k}}}
            \nonumber\\
        &= \prod_{k=0}^{t}
            \brackets{\prod_{P_{i}\in\mathcal{P}} C_{ik}}^{j^{k}}
    \label{eq:gpkj_rearrange}
\end{align}

\noindent
With this rearrangement, we see that inside the brackets we must perform
$t$ additions; this is followed by one multiplication.
This set of operations is performed $t+1$ times.
To combine these results also requires $t$ more additions.
In all, we have $t^{2} + 2t$ \textsc{ECAdd} operations,
$t+1$ \textsc{ECMul} operations,
and 1 \textsc{PairingCheck} operation.
Given that $t \sim \frac{2}{3}n$, we have

\begin{equation}
    \text{Cost of Proof} \sim 113000 + 4200n + 67n^{2}.
\end{equation}

\noindent
For $n=20$, this corresponds to $224$K gas;
the gas limit is approximately 10M.
We note, though, that this does not contain gas costs related
to verifying the submitted data, which is nontrivial.



\paragraph{Accusation Discussion}
It is important that the total number of validators be limited
so that the accusation logic does not become too large.
It is \emph{critical} that the accusation transaction
be easily formed and executed in Ethereum in order to ensure
a quick response to malicious behavior.
Furthermore, it is not only essential that the total gas
be within the limit, but also ensure that it will be processed
\emph{quickly} in order to ensure that all accusations are
processed within the necessary block window.

If a malicious actor submits an invalid $\gpk_{j}$,
this does \emph{not} affect the ability of the rest of the validators
to form a valid group signature.
It would, though, require the byzantine fault-tolerant majority
to move proceed.
We would prefer for this situation to never arise, though.



\subsubsection{\Completion{}}
\label{sssec:gpkj_complete}
At this point, the master public key is stored
and the group public keys are correct.
The DKG protocol is complete.

In our case, we confirm that DKG has been completed in order
for MadNet to proceed.

