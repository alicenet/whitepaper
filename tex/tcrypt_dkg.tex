\subsection{Distributed Key Generation Protocol}
\label{ssec:dkg}

We desire a Byzantine Fault Tolerant consensus algorithm.
So, we let $\mathcal{P}$ be the total collection of participants with
$\abs{\mathcal{P}} = n$.
We set the threshold $t = \floor{2n/3}$ in our $\parens{t,n}$
secret sharing protocol.
Thus, it takes $t+1$ users to reconstruct a secret, which corresponds
to strictly greater than two-thirds of the participants.
We assume there is an open broadcast channel between all participants.
Encryption will be provided through Diffie-Hellman-style shared
secret encryption; this will be discussed at the end of
Sec.~\ref{sssec:share_distribution}.
The group's shared secret, henceforth called the \emph{master secret key},
will be the sum of the shared secrets of
each group member who correctly shared his secret.
Once there are $t+1$ valid partial signatures, these will be
combined to form a group signature.

As stated above, although the final master public key will
reside in $\G_{2}$, because the precompiled contracts
currently available in the Ethereum Virtual Machine only allow addition
and scalar multiplication in $\G_{1}$ (multiplication and exponentiation
in our multiplicative notation), we will primarily
use computations in $\G_{1}$ and anything required
in $\G_{2}$ will be confirmed via a \textsc{PairingCheck} call.

\subsubsection*{Protocol Overview}

We briefly give an overview of the DKG protocol.
Here are the different phases of the protocol:

\begin{enumerate}
\item \textbf{\Registration{}:}
    Participants submit a public key
    which will allow for secure communication on an open channel.
\item \textbf{\ShareDistribution{}:}
    Secret shares are distributed
    through verifiable secret sharing;
    these shares allow for the construction of the master public key.
\item \textbf{\ShareDistributionDispute{}:}
    If there are any invalid shares,
    accusations can be made against malicious validators.
\item \textbf{\KeyShare{}:}
    Portions of the master public key and accompanying
    proofs are submitted.
\item \textbf{\MasterPublicKey{}:}
    Any participant is able to construct and submit
    the master public key.
\item \textbf{\GPKSubmission{}:}
    Group public keys are submitted.
\item \textbf{\GPKDispute{}:}
    Accusations are made against those who submitted invalid
    group public keys.
\item \textbf{\Completion{}:}
    Any participant is able to signal the successful completion
    of the DKG protocol.
    This is required for MadNet to proceed.
\end{enumerate}



\subsubsection{\Registration{}}
Each participant $P_{i}\in\mathcal{P}$ begins by selecting
a secret key $\sk_{i}\in\Z_{q}$ with public key $\pk_{i} = g_{1}^{\sk_{i}}$.
The public-private key pair $\angles{\pk_{i},\sk_{i}}$ will be
used for secure communication over the insecure broadcast channel;
it will not be used for signing any messages.
The public key $\pk_{i}$ will be submitted during registration.



\subsubsection{\ShareDistribution{}}
\label{sssec:share_distribution}
The primary purpose of this phase is to correctly share secrets
between validators.
These shared secret will then be used to establish a master secret key.
By distributing the appropriate shares,
each participant will be able to construct a portion
of the key to create group signatures.
This is done via Verifiable Secret Sharing
and is based on a distributed version of
Shamir's Secret Sharing~\cite{shamir1979share}.

Participant $P_{i}$ chooses a secret
$s_{i}\in\Z_{q}$ to share with the other participants.
To do this, choose a secret polynomial $f_{i}:\Z_{q}\to\Z_{q}$
with

\begin{equation}
    f_{i}(x) = c_{i0} + c_{i1}x + c_{i2}x^{2} + \cdots + c_{it}x^{t},
\end{equation}

\noindent
where $c_{i0} = s_{i}, c_{i1}, \cdots, c_{it}$ are chosen uniformly
in $\Z_{q}$.
Setting

\begin{equation}
    C_{ik} = g_{1}^{c_{ik}}\quad k\in\braces{0,\cdots,t},
\end{equation}

\noindent
we have the corresponding public polynomial $F_{i}:\Z_{q}\to\G_{1}$:

\begin{equation}
    F_{i}(x) = C_{i0}C_{i1}^{x}\cdots C_{it}^{x^{t}}.
\end{equation}

\noindent
The shared secret from $P_{i}$ to $P_{j}$ is $s_{i\to j} = f_{i}(j)$
and

\begin{equation}
    \overline{\texttt{s}}_{i\to j} =
        \textsc{Encrypt}(\text{sk}_{i},\text{pk}_{j},j, s_{i\to j})
\end{equation}

\noindent
refers to a particular encryption scheme discussed
at the end of this \hyperref[par:secret_enc]{section}.
Participant $P_{i}$ will broadcast the message

\begin{equation}
    \left\{ 
        \overline{\texttt{s}}_{i\to 1}, \overline{\texttt{s}}_{i\to 2},
            \cdots,
            \overline{\texttt{s}}_{i\to i-1},
            \overline{\texttt{s}}_{i\to i+1},
            \cdots,
            \overline{\texttt{s}}_{i\to n},
        C_{i0}, C_{i1}, \cdots, C_{it}
    \right\}
\end{equation}

\noindent
over the broadcast channel.
In our setting, this information will be submitted to the
Ethereum smart contract and
all participants will receive notification.
We note this message does not include the secret
$\overline{\texttt{s}}_{i\to i}$.

Once participant $P_{j}$ receives the message from $P_{i}$,
he sets

\begin{equation}
    \hat{s}_{i\to j} = \textsc{Decrypt}(\text{sk}_{j},\text{pk}_{i}, j,
        \overline{\texttt{s}}_{i\to j}).
\end{equation}

\noindent
$P_{j}$ then determines if

\begin{equation}
    g_{1}^{\hat{s}_{i\to j}} \overset{?}{=} F_{i}(j).
    \label{eq:secret_share_test}
\end{equation}

\noindent
If we have equality, then $\hat{s}_{i\to j} = s_{i\to j}$.
Otherwise, $P_{i}$ incorrectly shared his secret.

\paragraph{Shared Secret Encryption}
\label{par:secret_enc}

As mentioned above, the shared secret from
$P_{i}$ to $P_{j}$ is $s_{i\to j} = f_{i}(j)$.
To encrypt this, we need to compute their shared secret:

\begin{equation}
    k_{ij} = \pk_{i}^{\sk_{j}} = \pk_{j}^{\sk_{i}} =
    g_{1}^{\sk_{i}\sk_{j}}.
\end{equation}

\noindent
Encryption and decryption are based on the idea of a one-time pad;
in particular, we use outputs of cryptographic hashes of
the $x$-coordinate of the shared secret along with the index
of the participant receiving the message as our ``one-time pad''.
This does not meet the technical definition of a one-time pad
as it is usually defined (one standard reference is~\cite{hac1996}),
but the idea is similar.
By including the index of the intended recipient in the hash function,
each symmetric encryption key is unique.
See Alg.~\ref{alg:enc_dec} for details.

\begin{algorithm}[t]
\caption{Encryption and decryption functions}
\label{alg:enc_dec}
\begin{algorithmic}[1]
\Function{Encrypt}{$\sk$,$\pk$,$j$,$s$}
    \Comment{Encrypt secret $s$ to participant $j$;
        $\pk$ is $j$'s public key}
    \State $k = \pk^{\sk}$
    \Comment{k is the shared secret}
    \State $\texttt{kX}  = \texttt{u2b}(k_{x})$
    \Comment{Convert $x$ coordinate of shared secret to bytes}
    \State $\texttt{j}   = \texttt{u2b}(j)$
    \State $\texttt{s}   = \texttt{u2b}(s)$
    \State $\texttt{HKj} = H(\texttt{kX} || \texttt{j})$
    \State $\overline{\texttt{s}} = \texttt{s} \oplus \texttt{HKj}$
    \State \Return $\overline{\texttt{s}}$
\EndFunction

\State
\Function{Decrypt}{$\sk$,$\pk$,$j$,$\overline{\texttt{s}}$}
    \Comment{Decrypt secret $\overline{\texttt{s}}$ to participant $j$;
        $\sk$ is $j$'s secret key}
    \State $k = \pk^{\sk}$
    \Comment{k is the shared secret}
    \State $\texttt{kX}  = \texttt{u2b}(k_{x})$
    \Comment{Convert $x$ coordinate of shared secret to bytes}
    \State $\texttt{j}   = \texttt{u2b}(j)$
    \State $\texttt{HKj} = H(\texttt{kX} || \texttt{j})$
    \State $\texttt{s}   = \overline{\texttt{s}} \oplus \texttt{HKj}$
    \State $s            = \texttt{b2u}(\texttt{s})$
    \State \Return $s$
\EndFunction

\State
\Function{DecryptSS}{$k$,$j$,$\overline{\texttt{s}}$}
    \Comment{Decrypt secret $\overline{\texttt{s}}$ to participant $j$}
    \State $\texttt{kX}  = \texttt{u2b}(k_{x})$
    \Comment{Convert $x$ coordinate of shared secret to bytes}
    \State $\texttt{j}   = \texttt{u2b}(j)$
    \State $\texttt{HKj} = H(\texttt{kX} || \texttt{j})$
    \State $\texttt{s}   = \overline{\texttt{s}} \oplus \texttt{HKj}$
    \State $s            = \texttt{b2u}(\texttt{s})$
    \State \Return $s$
\EndFunction
\end{algorithmic}
\end{algorithm}




\subsubsection{\ShareDistributionDispute{}}
If we have $g_{1}^{\hat{s}_{i\to j}} = F_{i}(j)$
in Eq.~\eqref{eq:secret_share_test} for all participants,
then everyone correctly shared his secrets.
In this case, there is nothing to dispute and the DKG protocol
will proceed to the next phase.

Instead, we now suppose that $\overline{\texttt{s}}_{i\to j}$ is incorrect;
that is, $g_{1}^{\hat{s}_{i\to j}} \ne F_{i}(j)$
in Eq.~\eqref{eq:secret_share_test} for some $i$ and $j$.
In order to prove this to be the case, everyone needs to be
able to prove that the encrypted secret
$\overline{\texttt{s}}_{i\to j}$ is incorrect.
To do this, $P_{j}$ must publish and prove the shared secret $k_{ij}$;
this is required in order to ensure bad actors
do not submit false proofs against honest actors.

Proving $k_{ij}$ is the shared secret is based on showing

\begin{equation}
    \pk_{j} = g_{1}^{\sk_{j}} \quad\text{and}\quad
    k_{ij} = \pk_{i}^{\sk_{j}}
\end{equation}

\noindent
\emph{without} sharing the secret key $\sk_{j}$;
that is, we wish to show
$\dlog_{g_{1}}(\pk_{j}) = \dlog_{\pk_{i}}(k_{ij})$
while keeping their common value ($P_{j}$'s secret key $\sk_{j}$) secret.
To do this, we use a zero-knowledge proof;
see Alg.~\ref{alg:zk_dleq_proof} for constructing the zk-proof
and Alg.~\ref{alg:zk_dleq_verify} for proof verification.
One reference for zk-proofs involving discrete logarithms
is~\cite{camenisch1997proof}.

Thus, $P_{j}$ would compute

\begin{equation}
    \pi(k_{ij}') = \textsc{DLEQ}(g_{1},\pk_{j},\pk_{i},k_{ij}',\sk_{j})
\end{equation}

\noindent
and publish $\angles{k_{ij}',\pi(k_{ij}')}$, where $k_{ij}'$
is claimed shared secret.
This allows anyone to use \textsc{DLEQ-verify} to determine
its validity.
If

\begin{equation}
    \textsc{DLEQ-verify}(g_{1},\pk_{j},\pk_{i},k_{ij}',\pi(k_{ij}'))
        = \texttt{true},
\end{equation}

\noindent
then $k_{ij}' = k_{ij}$, the shared secret.
Using this, everyone can decrypt $\overline{\texttt{s}}_{i\to j}$
by

\begin{equation}
    \hat{s}_{i\to j}
        = \textsc{DecryptSS}(k_{ij},j,\overline{\texttt{s}}_{i\to j}, b)
\end{equation}

\noindent
and determine if

\begin{equation}
    g_{1}^{\hat{s}_{i\to j}} \overset{?}{=} F_{i}(j).
    \label{eq:secret_share_test_2}
\end{equation}

\noindent
If the DLEQ proof $\pi(k_{ij}')$ shows $k_{ij}'$ is the shared
secret between $P_{j}$ and $P_{i}$ and we do not have equality in
Eq.~\eqref{eq:secret_share_test_2}, then $P_{i}$ is acted maliciously
and should be removed.
There are two other possibilities:
$k_{ij}'$ is not the shared secret, or $k_{ij}'$ is the shared secret
and we have equality in Eq.~\eqref{eq:secret_share_test_2}.
In both cases, $P_{j}$ acted maliciously and should be removed.
Thus, when $P_{j}$ submits a claim that $P_{i}$
failed to share a secret,
either $P_{j}$'s or $P_{i}$'s stake will be slashed.

In practice, $P_{j}$ will submit $P_{i}$'s broadcast message to an
Ethereum smart contract along with purported shared secret $k_{ij}'$
and proof $\pi(k_{ij}')$, and the smart contract would
determine its validity and burn stake as appropriate.

\begin{algorithm}[p]
    \caption{Zero-knowledge proof of discrete-logarithm equality.}
\label{alg:zk_dleq_proof}
\begin{algorithmic}[1]
\Function{DLEQ}{$x_{1}$,$y_{1}$,$x_{2}$,$y_{2}$,$\alpha$}
    \Comment{Construct zk-proof that $y_{1} = x_{1}^{\alpha}$
        and $y_{2} = x_{2}^{\alpha}$.}
    \State $w\in_{R}\Z_{q}$
    \Comment{$x_{i}, y_{i}\in\G_{1}$ with $\abs{\G_{1}}=q$.}
    \State $t_{1} = x_{1}^{w}$
    \State $t_{2} = x_{2}^{w}$
    \State $\texttt{x1} = \texttt{u2b}(x_{1})$
    \State $\texttt{y1} = \texttt{u2b}(y_{1})$
    \State $\texttt{x2} = \texttt{u2b}(x_{2})$
    \State $\texttt{y2} = \texttt{u2b}(y_{2})$
    \State $\texttt{t1} = \texttt{u2b}(t_{1})$
    \State $\texttt{t2} = \texttt{u2b}(t_{2})$
    \State $\texttt{c}  = H(\texttt{x1}||\texttt{y1}||\texttt{x2}||\texttt{y2}||\texttt{t1}||\texttt{t2})$
    \State $c = \texttt{b2u}(\texttt{c})$
    \State $r = w - \alpha c \mod q$
    \State $\pi = \parens{c,r}$
    \State \Return $\pi$
\EndFunction
\end{algorithmic}
\end{algorithm}

\begin{algorithm}[p]
    \caption{Zero-knowledge verification of discrete-logarithm equality proof.}
\label{alg:zk_dleq_verify}
\begin{algorithmic}[1]
\Function{DLEQ-verify}{$x_{1}$,$y_{1}$,$x_{2}$,$y_{2}$,$\pi$}
    \Comment{Determine validity of proof from \texttt{DLEQ}}
    \State $\parens{c,r} = \pi$
    \State $t_{1}' = x_{1}^{r}y_{1}^{c}$
    \State $t_{2}' = x_{2}^{r}y_{2}^{c}$
    \State $\texttt{x1}  = \texttt{u2b}(x_{1})$
    \State $\texttt{y1}  = \texttt{u2b}(y_{1})$
    \State $\texttt{x2}  = \texttt{u2b}(x_{2})$
    \State $\texttt{y2}  = \texttt{u2b}(y_{2})$
    \State $\texttt{t1p} = \texttt{u2b}(t_{1}')$
    \State $\texttt{t2p} = \texttt{u2b}(t_{2}')$
    \State $\texttt{cp}  = H(\texttt{x1}||\texttt{y1}||\texttt{x2}||\texttt{y2}||\texttt{t1p}||\texttt{t2p})$
    \State $c' = \texttt{b2u}(\texttt{cp})$
    \If {$c = c'$}
        \State \Return \texttt{true}
    \Else
        \State \Return \texttt{false}
    \EndIf
\EndFunction

\end{algorithmic}
\end{algorithm}




\subsubsection{\KeyShare{}}

At this point, we know that every individual correctly shared his secret.
Thus, we can now use this to submit the information required
to build the master public key.

Participant $P_{i}$'s secret $s_{i}$ is concealed behind
their commitment $C_{i0} = g_{1}^{s_{i}}$.
As discussed in~\cite{gennaro3revisiting,gennaro1999secure,ethdkg},
in order to ensure that no bad actors gain any information
about the master public key and are not be able to change
its underlying probability distribution, we require
$h_{1}\in\G_{1}$ such that $\dlog_{g_{1}}h_{1}$ is unknown.
We also let $h_{2}\in\G_{2}$ be a generator.

Let

\begin{equation}
    \pi(h_{1}^{s_{i}}) = \textsc{DLEQ}(
        g_{1},g_{1}^{s_{i}},h_{1},h_{1}^{s_{i}},s_{i})
\end{equation}

\noindent
be the zk-proof that $h_{1}^{s_{i}}$ is $P_{i}$'s portion
of the master public key (technically part of $\text{mpk}^{*}$
as defined below).
Because $C_{i0} = g_{1}^{s_{i}}$ is public knowledge
(stored in the smart contract)
and $P_{i}$ correctly shared his secret $s_{i}$,
the zk-proof will ensure the submitted value is valid.
Additionally, $P_{i}$ will publish $h_{2}^{s_{i}}$ so that
we can ensure

\begin{equation}
    \textsc{PairingCheck}(h_{1}^{s_{i}},\bar{h}_{2},h_{1},h_{2}^{s_{i}})
        = \texttt{true}.
\end{equation}

\noindent
This \textsc{PairingCheck} will be called by a smart contract.
Thus, failure of $P_{i}$ to publish $h_{1}^{s_{i}}$,
a valid proof $\pi(h_{1}^{s_{i}})$, and the corresponding
$h_{2}^{s_{i}}$ amounts to misbehavior which will
lead to a fine.
In particular, $P_{i}$ will submit
$\angles{h_{1}^{s_{i}}, \pi(h_{1}^{s_{i}}), h_{2}^{s_{i}}}$
to the smart contract.
The Ethereum smart contract will store $h_{1}^{s_{i}}$ from
all participants and broadcast
$h_{1}^{s_{i}}$, $\pi(h_{1}^{s_{i}})$, and $h_{2}^{s_{i}}$.

If any participant does not correctly submit these value,
then this entire process is required.



\subsubsection{\MasterPublicKey{}}

In this phase, a participant must submit the master public key.

At this phase, all participants have correctly shared
the secret share $s_{i}$ as well as $h_{1}^{s_{i}}$ and $h_{2}^{s_{i}}$.
The individual shared secrets $s_{i}$ allow us to define the
\emph{master secret key} $\text{msk}$:

\begin{equation}
    \text{msk} = \sum_{P_{i}\in\mathcal{P}} s_{i}.
\end{equation}

\noindent
This gives us the \emph{master public key} $\text{mpk}$:

\begin{align}
    \text{mpk} &= h_{2}^{\text{msk}} \nonumber\\
        &= \prod_{P_{i}\in\mathcal{P}} h_{2}^{s_{i}}.
\end{align}

\noindent
A Byzantine-fault tolerant subgroup $\mathcal{R}\subseteq\mathcal{P}$ can
correctly obtain the secret $s_{i}$ via Lagrange interpolation:

\begin{align}
    s_{i} &= \sum_{P_{j}\in\mathcal{R}} s_{i\to j} R_{j} \nonumber\\
    R_{j} &= \prod_{\substack{P_{k}\in\mathcal{R} \\ k\ne j}} \frac{k}{k-j}.
    \label{eq:Rj_coefs}
\end{align}

\noindent
This would allow us to recover the secret $s_{i}$ should $P_{i}$
fail to share $h_{1}^{s_{i}}$ below; however,
we take a stricter response and would view failure to share
as malicious activity leading to stake slashing.
Furthermore, the DKG process must restart.
We note that the Lagrange interpolation just described
occurs in a finite field;
this means that $R_{j}$ is constructed using finite field divisions.

Because $h_{2}^{s_{i}}$ has been broadcast to all participants,
anyone will be able to submit

\begin{equation}
    \text{mpk} = \prod_{P_{i}\in\mathcal{P}} h_{2}^{s_{i}}
\end{equation}

\noindent
to the smart contract.
Because $\braces{h_{1}^{s_{i}}}_{i\in\mathcal{P}}$ are stored
and valid, the contract can construct

\begin{equation}
    \text{mpk}^{*} = \prod_{P_{i}\in\mathcal{P}} h_{1}^{s_{i}}
    \label{eq:mpk_dual}
\end{equation}

\noindent
and confirm its validity with a \textsc{PairingCheck}
with the follow operation:

\begin{equation}
    \textsc{PairingCheck}(\text{mpk}^{*},\bar{h}_{2},h_{1},\text{mpk})
        = \texttt{true}.
\end{equation}

\noindent
The master public key can then be stored publicly and used
for group signature verification.



\subsubsection{\GPKSubmission{}}

At this point, we have successfully constructed the master public
key $\text{mpk}$ and distributed
the master secret key $\text{msk}$ among the members of $\mathcal{P}$.
We now turn our attention to constructing group
signatures from partial signatures.

Each participant $P_{j}\in\mathcal{P}$ has a portion of the
master secret key; this portion is called the \emph{group secret key}
$\gsk_{j}$:

\begin{equation}
    \gsk_{j} = \sum_{P_{i}\in\mathcal{P}} s_{i\to j}.
\end{equation}

\noindent
This is possible because we proved that every participant
in $\mathcal{P}$ correctly shared his secret share.
We note that that $s_{j\to j}$ is included in the sum for
$\gsk_{j}$ even though the encrypted form was not publicly shared.
Naturally, there is the corresponding \emph{group public key}
$\gpk_{j}$:

\begin{equation}
    \gpk_{j} = h_{2}^{\gsk_{j}}.
\end{equation}

\noindent
Here, $\gpk_{j}$ is $P_{j}$'s portion of the master public key
and will be broadcast to all users.
Cryptographic proof that $\gpk_{j}$ is valid will be discussed
in Sec.~\ref{app:gpkj_dispute}.

Each participant must submit group public key $\gpk_{j}$
and signature $\sigma_{j}$ for message $M$.
This signature is computed by

\begin{equation}
    \sigma_{j} = \brackets{H_{2C}(M)}^{\gsk_{j}}
\end{equation}

\noindent
with $\sigma_{j}\in\G_{1}$ and $M$ is a predetermined message to be signed.
Here, $H_{2C}:\braces{0,1}^{*}\to\G_{1}$ is a hash-to-curve function.
Hash-to-curve functions will be discussed in greater detail in
Sec.~\ref{ssec:hash-to-curve}.

During submission of $\angles{\gpk_{j},\sigma_{j}}$,
we confirm that

\begin{equation}
    \textsc{PairingCheck}(\sigma_{j},\bar{h}_{2},H_{2C}(M),\gpk_{j})
        = \texttt{true}.
\end{equation}

\noindent
This is performed to ensure that the signature is valid.
While this by itself does not ensure $\gpk_{j}$ is valid,
it helps enable some $\gpk_{j}$ validation logic.



\subsubsection{\GPKDispute{}}
\label{app:gpkj_dispute}

We now look at how to ensure the broadcast value of
$\gpk_{j}$ is valid.

Along with the $P_{j}$'s group public key $\gpk_{j}\in\G_{2}$,
there is the corresponding dual version in $\G_{1}$:

\begin{align}
    \gpk_{j}^{*} &= g_{1}^{\gsk_{j}}
            \nonumber\\
        &= \prod_{P_{i}\in\mathcal{P}}F_{i}(j).
    \label{eq:gpkj_star_def}
\end{align}

\noindent
Note the base is $g_{1}$ and not $h_{1}$ as in the case of $\text{mpk}^{*}$
in Eq.~\eqref{eq:mpk_dual}.
Here, we remember that

\begin{equation}
    F_{i}(j) = C_{i0}C_{i1}^{j}C_{i2}^{j^{2}}\cdots C_{it}^{j^{t}}.
    \label{eq:Fij_def}
\end{equation}

\noindent
This shows that $\gpk_{j}^{*}$ is public knowledge
and can be used to determine validity;
that is, we form $\gpk_{j}^{*}$ locally and the check that

\begin{equation}
    \textsc{PairingCheck}(\gpk_{j}^{*}, \bar{h}_{2}, g_{1}, \gpk_{j})
        = \texttt{true}.
\end{equation}

\noindent
If all $\gpk_{j}$ submissions are valid,
then there is nothing to dispute and we can proceed to the next phase.
Otherwise, some accusations must be made to prove malicious behavior.

\paragraph{Gas Cost Discussion}
The major costs will be calling the precompiled contracts
\textsc{ECAdd}, \textsc{ECMul}, \textsc{PairingCheck},
and \textsc{ModExp} (modular exponentiation);
see Table~\ref{tab:evm_gas_cost} for the specific gas costs.
Note that the cost for \textsc{PairingCheck} comes
from our assumption that we are testing 2 pairings,
while the cost for \textsc{ModExp} comes from the fact
that all of our arguments are 256-bit (32-byte) unsigned integers.
These costs come from EIP-198\footnote{
    \url{https://github.com/ethereum/EIPs/blob/master/EIPS/eip-198.md}}
and EIP-1108\footnote{
    \url{https://github.com/ethereum/EIPs/blob/master/EIPS/eip-1108.md}}.

\begin{table}
\centering
\begin{tabular}{|c|c|}
\hline
Precompile & Gas Cost \\
\hline
\hline
\textsc{ECAdd} & 150 \\
\textsc{ECMul} & 6000 \\
\textsc{PairingCheck} &  113000 \\
\hline
\end{tabular}
\caption[EVM Gas Cost]{Gas cost of important precompiled contracts on the
    Ethereum Virtual Machine.}
\label{tab:evm_gas_cost}
\end{table}





\paragraph{GPK Standard Accusation}
The standard accusation is when we submit the necessary information
to compute $\gpk_{j}^{*}$ to a smart contract and then show
that the submission fails the pairing check.

Using Eqs.~\eqref{eq:gpkj_star_def} and \eqref{eq:Fij_def},
we see

\begin{align}
    \gpk_{j}^{*} &= \prod_{P_{i}\in\mathcal{P}}F_{i}(j)
            \nonumber\\
        &= \prod_{P_{i}\in\mathcal{P}} \brackets{\prod_{k=0}^{t} C_{ik}^{j^{k}}}
            \nonumber\\
        &= \prod_{k=0}^{t}
            \brackets{\prod_{P_{i}\in\mathcal{P}} C_{ik}}^{j^{k}}
    \label{eq:gpkj_rearrange}
\end{align}

\noindent
With this rearrangement, we see that inside the brackets we must perform
$t$ additions; this is followed by one multiplication.
This set of operations is performed $t+1$ times.
To combine these results also requires $t$ more additions.
In all, we have $t^{2} + 2t$ \textsc{ECAdd} operations,
$t+1$ \textsc{ECMul} operations,
and 1 \textsc{PairingCheck} operation.
Given that $t \sim \frac{2}{3}n$, we have

\begin{equation}
    \text{Cost of Standard Proof} \sim 113000 + 4200n + 67n^{2}.
\end{equation}

\noindent
For $n=20$, this corresponds to $224$K gas;
the gas limit is approximately 10M.



\paragraph{GPK Group Accusation}
The group accusation is when we form a group signature and
show that using $P_{j}$'s submitted information results
in an invalid signature.
If we have a Byzantine fault-tolerant majority,
this shows that $\gpk_{j}$ is invalid.

To form a group signature, we must compute

\begin{equation}
    \sigma = \prod_{P_{j}\in\mathcal{R}} \sigma_{j}^{R_{j}},
\end{equation}

\noindent
where $\mathcal{R}\subseteq\mathcal{P}$ is the set of participants
who form a group signature.
We must have $\abs{\mathcal{R}} = t+1$.
From there, we verify the signature with the following operation:

\begin{equation}
    \textsc{PairingCheck}(\sigma, \bar{h}_{2}, H_{2C}(M), \text{mpk})
        = \texttt{true}.
\end{equation}

\noindent
Here, $M$ is publicly known and $\text{mpk}$ has already
been verified.

Each group signature $\sigma$ will require forming $R_{j}$
and computing $\sigma_{j}^{R_{j}}$.
After computing $R_{j}$, we require one \textsc{ECMul} call
to compute $\sigma_{j}^{R_{j}}$.
The expensive part turns out to be forming $R_{j}$,
and we focus on its straightforward computation.
We recall from Eq.~\eqref{eq:Rj_coefs} that

\begin{equation}
    R_{j} = \prod_{\substack{P_{k}\in\mathcal{R} \\ k\ne j}} \frac{k}{k-j}.
\end{equation}

\noindent
Naively, we would need to compute $t$ finite field inversions,
which corresponds to $t$ \textsc{ModExp} calls, to compute $R_{j}$;
this must be performed $t+1$ times to form $\sigma$.
This results in an expensive operation.

To reduce the required gas cost, we precompute these
multiplicative inverses and include them in the function call.
At the beginning of the call, we check to make sure that the submitted
inverses are valid; if they are valid, we proceed with the
accusation, and if they are invalid, we stop.
Everything else is the same as before;
see Alg.~\ref{alg:grpsig_malicious} for the complete description.
For each accusation, the main cost is $t+1$ \textsc{ECMul} calls,
$t$ \textsc{ECAdd} calls,
and one pairing check.
Therefore, we see

\begin{equation}
    \text{Cost to Form and Verify Group Signature} \sim 113000 + 4100n.
\end{equation}

\noindent
For $n=20$, this corresponds to $195$K gas.



\paragraph{Accusation Discussion}
In order to allow for a valid comparison between the two methods,
we must realize that any proof of malicious behavior involving
group signatures must require an \emph{additional} group signature
construction to form a valid group signature.
As we can see from Table~\ref{tab:grpsig_accusation_cost},
the more efficient accusation method depends upon
the number of malicious actors as well as the total validator set.

It is important that the total number of validators be limited
so that the accusation logic does not become too large.
It is \emph{critical} that the accusation transaction
be easily formed and executed in Ethereum in order to ensure
a quick response to malicious behavior.

If a malicious actor submits an invalid $\gpk_{j}$,
this does \emph{not} affect the ability of the rest of the validators
to form a valid group signature.
It would, though, require the byzantine fault-tolerant majority
to move proceed.

\begin{table}[t]
\centering
\begin{tabular}{|c||c|c||c|c||c|c|}
\hline
& \multicolumn{6}{|c|}{Group Accusation Gas Cost} \\
\hline
\hline
$m$ & \multicolumn{2}{|c||}{$n = 10$} & \multicolumn{2}{|c||}{$n = 20$} &
    \multicolumn{2}{|c|}{$n = 30$} \\
\hline
    1 &  162K & 308K &  224K &  390K &  299K &  472K \\
    2 &  323K & 462K &  448K &  585K &  599K &  708K \\
    3 &  485K & 616K &  671K &  780K &  898K &  994K \\
    4 &  647K &      &  895K &  975K & 1.20M & 1.18M \\
    5 &  808K &      & 1.12M & 1.17M & 1.50M & 1.42M \\
    6 &  970K &      & 1.34M & 1.36M & 1.80M & 1.65M \\
    7 & 1.13M &      & 1.57M &       & 2.10M & 1.89M \\
\hline
\end{tabular}
\caption[Gas Cost for Invalid Signature]{
    Here we list the gas cost for performing group accusation
    for both the standard and group signature method.
    As we can see, which method is more efficient depends both on
    the total validator set ($n$) as well as the number of malicious
    validators ($m$).
    }
\label{tab:grpsig_accusation_cost}
\end{table}


\begin{algorithm}[p]
\caption{Accusation against malicious $\gpk_{j}$ shares using
    group signatures.}
\label{alg:grpsig_malicious}
\begin{algorithmic}[1]
\Function{GrpAccGPKj}{$\textsc{invArray}$, $\textsc{honestIndices}$,
        $\textsc{dishonestIndices}$}
    \If{$\textsc{Length}(\textsc{honestIndices}) < t+1$}
        \State \Return
        \Comment{Require $t+1$ validators to make a valid signature}
    \EndIf
    \State \texttt{validInvs} = \textsc{CheckInverses}(\textsc{invArray})
    \If{$\texttt{validInvs} \ne \texttt{true}$}
        \State \Return
        \Comment{Did not submit valid multiplicative inverses}
    \EndIf
    \State $\textsc{IndexArray} = \textsc{honestIndices}[0:t+1]$
    \Comment{Include first $t+1$ participants}
    \State $\sigma = \textsc{AggregateSignatures}(\textsc{IndexArray})$
    \State $\texttt{validSig} = \textsc{PairingCheck}(\sigma, \bar{h}_{2},
        H_{2C}(M), \text{mpk})$
    \If{$\texttt{validSig} \ne \texttt{true}$}
        \State \Return
        \Comment{\textsc{honestIndices} do not form a valid group signature}
    \EndIf
    \For{$i=0$; $i < \textsc{Length}(\textsc{dishonestIndices})$; $i$++}
        \State $\textsc{IndexArray}[t]
            = \textsc{dishonestIndices}[i]$
        \State $\sigma = \textsc{AggregateSignatures}(\textsc{IndexArray})$
        \State $\texttt{validSig} = \textsc{PairingCheck}(\sigma, \bar{h}_{2},
            H_{2C}(M), \text{mpk})$
        \If{$\texttt{validSig} \ne \texttt{false}$}
            \State \Return
            \Comment{$\textsc{dishonestIndices}[i]$ submitted valid
                signature}
        \EndIf
    \EndFor
\EndFunction
\end{algorithmic}
\end{algorithm}

\begin{algorithm}[p]
\caption{Determine if submitted inverses are correct}
\label{alg:grpsig_mal_Rj}
\begin{algorithmic}[1]
\Function{CheckInverses}{\textsc{invArray}}
    \State \texttt{validInvs} = \texttt{true}
    \Comment{Assuming the form $\brackets{1^{-1}, 2^{-1}, \cdots, (n-1)^{-1}}$}
    \For{$i=0$; $i < \textsc{Length}(\textsc{invArray})$; $i$++}
        \State $k = i+1$
        \State $k_{\text{inv}} = \textsc{invArray}[i]$
        \State $r = k_{\text{inv}}\cdot k \mod q$
        \If{$r\ne1$}
            \State \texttt{validInvs} = \texttt{false}
            \State \textbf{break}
        \EndIf
    \EndFor
    \State \Return \texttt{validInvs}
\EndFunction
\end{algorithmic}
\end{algorithm}

\begin{algorithm}[p]
\caption{Compute group signature}
\label{alg:compute_grpsig}
\begin{algorithmic}[1]
\Function{AggregateSignatures}{\textsc{IndexArray}, \textsc{invArray}}
    \State $\sigma = \textsc{Identity}$
    \Comment{Set to identity element of $\G_{1}$}
    \For{$\texttt{idx} = 0$; $\texttt{idx} < t+1$; $\texttt{idx}$++}
        \State $j = \textsc{IndexArray}[\texttt{idx}]$
        \State $R_{j} = \textsc{ComputeRj}(j, \textsc{IndexArray},
            \textsc{invArray})$
        \State $t = \textsc{ECMul}(\sigma_{j}, R_{j})$
        \Comment{$\sigma_{j}$ is $j$'s stored signature}
        \State $\sigma = \textsc{ECAdd}(\sigma,t)$
    \EndFor
    \State \Return $\sigma$
\EndFunction
\end{algorithmic}
\end{algorithm}

\begin{algorithm}[p]
\caption{Compute $R_{j}$ for group signature}
\label{alg:grpsig_compute_Rj}
\begin{algorithmic}[1]
\Function{ComputeRj}{$j$, \textsc{IndexArray}, \textsc{invArray}}
    \State $R_{j} = 1$
    \For{$\texttt{idx}=0$; $\texttt{idx}<\textsc{Length}(\textsc{IndexArray})$;
            $\texttt{idx}$++}
        \State $k = \textsc{IndexArray}[\texttt{idx}]$
        \State $t = k$
        \If{$k = j$}
            \State \textbf{continue}
        \ElsIf{$k > j$}
            \State $\alpha = k - j$
        \Else
            \State $t = (-1)\cdot t \mod q$
            \State $\alpha = j - k$
        \EndIf
        \State $t_{\text{inv}} = \textsc{invArray}[\alpha-1]$
        \State $t = t\cdot t_{\text{inv}} \mod q$
        \State $R_{j} = R_{j}\cdot t \mod q$
    \EndFor
    \State \Return $R_{j}$
\EndFunction
\end{algorithmic}
\end{algorithm}




\subsubsection{\Completion{}}

At this point, the master public key is stored
and the group public keys are correct.
The DKG protocol is complete.

In our case, we confirm that DKG has been completed in order
for MadNet to proceed.

