\section{Networking}

\subsection{Overview of Networking Stack}

The networking stack consists of several layers that have been selected
to allow for code generation and security.
These layers consist of low level networking protocols upon which the
Peer-to-Peer networking, Discovery, and BootNode protocol have been built.

\subsection{Brontide}

In order to provide authenticated encryption, we selected Brontide as
the backbone of our networking stack.
Brontide is the authenticated encryption protocol that is used by the
Bitcoin Lightning Network and has been used securely in the wild for some time.
The Brontide system was derived from the Noise Protocol Framework.
Noise uses Diffie-Hellman key agreement to enable Authenticated
Encryption with Associated Data (AEAD).
The Noise Protocol Framework was created by Trevor Perrin, one of the
original authors of the precursors to the Signal Protocol.
The official protocol name Brontide uses is

\begin{equation*}
\texttt{Noise\_XK\_secp256k1\_ChaChaPoly\_SHA256}.
\end{equation*}

\noindent
Here, the \texttt{Noise\_XK} handshake specifies the initiator knows the
responders static public key, so it is never transmitted.
The additional requirements include the elliptic curve \secp{}, the
AEAD method based on the ChaCha20 symmetric cipher and Poly1305
authenticator, and the cryptographic hash function SHA256.
We selected this type of protocol to ensure that a peer was protected
from MitM attacks if the peer was connecting to a previously known host.
Further, the protocol ensures data integrity without the need to depend
on x.509 systems.

\subsection{Yamux}

Yamux is a robust multiplexing library built by HashiCorp.
This lightweight TCP streaming multiplexor protocol was built to allow
a TCP connection to be converted into an ordered full duplex
communication channel that allows multiple logical connections to exist
over a single TCP connection.
This multiplexor allows gRPC to be used in our system seamlessly while
also only requiring a single connection between peers.

\subsection{gRPC}

gRPC is a mature remote procedure call system originally developed at
Google before being released to the public.
It uses Protobuf (Protocol Buffers) for data serialization which allows
for easy client-server communication necessary for the higher-order
protocols.
gRPC is used by many organizations including Netflix, Docker, and
Spotify and has proven capable in demanding applications.

\subsection{Connection Handshaking}

In addition to the connection handshaking associated with Brontide,
before any higher level protocol begins, two messages must be exchanged.
The first message exchanges the Chain ID of each peer.
If these values do not match, the connection is dropped.
This has been built into this level of the protocol to prevent problems
seen in other blockchain networks where peers attempt to sync the wrong chain.
We chose instead to validate both peers are on the same network and
terminate on failure as soon as possible.
The second message communicates the port on which the remote node may
be dialed back.
The specific port that is passed in this location is the port on which
the P2P protocol of a peer may be found.
The reason this value is passed at this level is to facilitate the
proper operation of discovery by forming a complete identity of a
remote node.
This identity includes the host, port, and public key of a remote
node

\subsection{Summary of Higher Order Protocols}

At this time there are three higher order protocols in the
communication stack.
These are the BootNode protocol, the Discovery protocol, and the P2P
protocol.
These protocols have been divided into these classifications to allow
differentiation of operations and security requirements.
The BootNode protocol allows a peer who is first joining the network to
discover peers in the overlay network using a persistent entry point.
The Discovery protocol allows a peer to be connected to another peer
for a single request that communicates other peers that a node may
connect to.
This protocol allows a node to build a peer list.
Finally, the P2P protocol is a persistent full duplex channel between
nodes in the P2P network.
All protocols are based on gRPC after the initial handshake.

\subsection{BootNode Protocol Summary}

The BootNode protocol operates by allowing a remote host to connect to
a persistent point of entry into the overlay P2P network.
The bootnode servers are ignorant of the block chain and serve the
singular purpose of allowing peering.
These servers hold a cache of long term peers that have remained
responsive to intermittent heartbeats over several days.
These servers also hold an LRU cache that contains the most recently
seen nodes up to cache eviction.
Any node that connects to a bootnode server will make a single request
that returns a list of peers that may be used for discovery and this
list will contain items from both of the above mentioned caches.
The bootnode server will terminate the connection after this single
request is served.

\subsection{Discovery Protocol Summary}

The Discovery Protocol allows a node to find more peers.
Peers are stored in one of sixteen buckets based on the first hex
character of the public key of the peer.
These buckets contain two sublists.
One list stores active P2P connections while the other list stores
known nodes from the discovery protocol requests.
The algorithm attempts to fill the set of not connected buckets by
dialing a random subset of peers from each bucket and requesting the
closest known peers to a random address.
This is similar in nature to how Kademlia operates, but not identical.
The returned peers are added to the list of possible peers to connect
with.
Once a node has enough peers that it may connect to, it begins dialing
a random peer from each bucket until a user-defined number of peers
have been connected to using the P2P protocol.

\subsection{P2P Protocol Summary}

The P2P protocol allows peers to gossip information about blockchain
state, share consensus messages, and gossip transactions.
This protocol is, like the other protocols, based on gRPC.
The only difference is that all connections under this protocol are
simulated connections that have been constructed from the Yamux
multiplexing protocol.
This allows the peers to act as both server and client in the P2P
network while also leveraging the robust nature of gRPC in the process.
