\subsection{Specific Implementation Details}
\label{ssec:pk_curve_specifics}

At this point in time, the Ethereum Virtual Machine (EVM) 
only allows for certain elliptic curve operations to
be performed inexpensively from precompiled contracts.
In order to keep gas costs low, we will rely heavily 
on these contracts.
The precompiled contracts are
\textsc{ECAdd}, \textsc{ECMul}, and \textsc{PairingCheck};
see Alg.~\ref{alg:ec_algs} for specifics.
While they have the names addition and scalar multiplication,
we will primarily be using multiplicative notation in this whitepaper;
the exception will be when we talk about hash-to-curve algorithms
in Sec.~\ref{ssec:hash-to-curve}.
Furthermore, we may not explicitly reference \textsc{ECAdd}
and \textsc{ECMul} when using them in our algorithms
for space considerations, although we will use comments to make
this clear.

\begin{algorithm}[t]
\caption{Precompiled Ethereum Contracts for Elliptic Curves}
\label{alg:ec_algs}
\begin{algorithmic}[1]
\Function{ECAdd}{$a_{1}$,$a_{2}$}
    \Comment{$a_{1},a_{2}\in\G_{1}$}
    \State \Return $a_{1}+a_{2}$
    \Comment{Elliptic curve addition}
\EndFunction

\State
\Function{ECMul}{$a$,$k$}
    \Comment{$a\in\G_{1}, k\in\F_{p}$}
    \State \Return $\brackets{k}a$
    \Comment{Elliptic curve addition}
\EndFunction

\State
    \Function{PairingCheck}{$a_{0}$,$b_{0}$,$\cdots$,$a_{k-1}$,$b_{k-1}$}
    \Comment{$a_{i}\in\G_{1}, b_{i}\in\G_{2}$}
    \State $t=1$
    \Comment{$t\in\F_{p^{12}}^{*}$}
    \For{$i=0$; $i<k$; $i$++}
        \State $t = t\cdot e\parens{a_{i},b_{i}}$
    \EndFor
    \If{$t=1$}
        \State \Return \texttt{true}
    \Else
        \State \Return \texttt{false}
    \EndIf
\EndFunction

\end{algorithmic}
\end{algorithm}


We require elliptic curves with pairing-based cryptography.
The source code for curve \texttt{bn256} is implemented
for Ethereum here\footnote{
\url{https://github.com/ethereum/go-ethereum/tree/master/crypto/bn256/cloudflare}
}.
This code implements the Optimal Ate pairing from~\cite{naehrig2010new} and
is based on Barreto-Naehrig curves~\cite{bnCurves},
a family of pairing-friendly curves.
A more recent version of Cloudflare's library can be found here\footnote{
\url{https://github.com/cloudflare/bn256}}.
Cloudflare's version includes a hash-to-curve function
based on~\cite{ft2012bnhashtocurve}, which is the same paper we base our
hash function on.
Unfortunately, the newer version uses a different prime;
the Ethereum library uses a 254-bit prime while the current Cloudflare
library uses a 256-bit prime.
We modified the Ethereum code
and included functions for computing modular square roots as
well as a hash-to-curve algorithm.
We plan to make these updates available to others, as they will
be useful whenever cryptographic signatures are used.
Our code includes an implementation for the hash-to-curve algorithms
described in~\cite{ft2012bnhashtocurve}; see Sec.~\ref{ssec:hash-to-curve}.

The underlying field operations perform arithmetic modulo a prime number.
This can be difficult to perform quickly~\cite[Chapter 14]{hac1996}.
The \texttt{bn256} library uses Montgomery
encoding~\cite{montgomery1985modular} for efficiency;
we will give an overview here although one reference
is~\cite[Chapter 14.3.2]{hac1996}.
The primary advantage of using the Montgomery encoding is that
modular multiplication requires only multiplication with a potential
subtraction; in particular, it does not require division.

If we want to perform multiplication modulo $p$, let $R>p$ such
that $\gcd(R,p) = 1$ and it is convenient to perform modular arithmetic
with respect to $R$.
Given $x\in\Z_{p}$, the Montgomery encoding of $x$ is

\begin{equation}
    \tilde{x} \equiv xR \mod p.
\end{equation}

\noindent
Given two encoded values $\tilde{a}$ and $\tilde{b}$,
Montgomery multiplication allows us to compute $\widetilde{ab}$:

\begin{equation}
    \widetilde{ab} = \tilde{a}\tilde{b}R^{-1} \mod p.
\end{equation}

\noindent
Here, $R^{-1}R = 1 \mod p$, so $R^{-1}$ is the multiplicative
inverse of $R$ with respect to $p$.
An efficient implementation of Montgomery multiplication
allows us to use it both for encoding and decoding,
and this is what the \texttt{bn256} library does.

We now look at the elliptic curves in the pairing-based cryptography.
Specifically, we have the elliptic curve $E/\F_{p}$ where

\begin{equation}
    E : y^{2} = x^{3} + ax + b
\end{equation}

\noindent
and constants

\begin{align}
    p &= \texttt{30644E72 E131A029 B85045B6 8181585D 97816A91 6871CA8D 3C208C16
                D87CFD47} \nonumber\\
        &= 36u^{4} + 36u^{3} + 24u^{2} + 6u + 1 \nonumber\\
    q &= \texttt{30644E72 E131A029 B85045B6 8181585D 2833E848 79B97091 43E1F593
                F0000001} \nonumber\\
        &= 36u^{4} + 36u^{3} + 18u^{2} + 6u + 1 \nonumber\\
    u &= 4965661367192848881 \nonumber\\
    a &= 0 \nonumber\\
    b &= 3 \nonumber\\
    g_{1} &= \parens{1,2} \nonumber\\
    h_{1} &= \parens{h_{1,x}, h_{1,y}} \nonumber\\
    h_{1,x} &= \texttt{081D36DE F693881E DFC5614E AE25BB5C 228A7142 A36EE533 47B09434 1D541F2C} \nonumber\\
    h_{1,y} &= \texttt{2CD20C36 14D407F3 39B9BB25 EF23979C D2EE1E45 310EB0C5 023A3F5F D52D8B11}
\end{align}

\noindent
From here, we see $p = 3 \mod 4$ and $p = 1 \mod 6$.
If $E(\F_{p})$ is the group of points on $E/\F_{p}$
acting under addition, then $E(\F_{p}) = \angles{g_{1}}$
and $\abs{E(\F_{p})} = q$; here, $g_{1}$ the standard generator.
From the construction of BN curves, $q$ is prime,
which implies that any nontrivial element of $\G_{1}$ is a generator.
During the distributed key generation protocol, we will
need another generator $h_{1}$ for $E(\F_{p})$ such that
$\dlog_{g_{1}}h_{1}$ is unknown, which ensures no one
is able to bias the underlying probability distribution
of the master public key.
Here, we set

\begin{equation}
    h_{1} = \textsc{HashToG1}(\texttt{[]byte("MadHive Rocks!")}),
\end{equation}

\noindent
where $\textsc{HashToG1}$ is the hash-to-curve function
from Alg.~\ref{alg:hash-to-G1} developed in
Sec.~\ref{ssec:hash-to-curve}.
In the future $h_{1}$ may change on each iteration of the DKG
protocol, but it is constant at this point.


We let $e:\G_{1}\times\G_{2}\to\G_{T}$ be our nondegenerate
bilinear map over groups $\G_{1}$, $\G_{2}$, and $\G_{T}$.
We let $\G_{1} = E(\F_{p})$, $\G_{2}\subseteq E(\F_{p^{12}})$,
and $\G_{T}\subseteq \F_{p^{12}}^{*}$;
$e$ is the Optimal Ate pairing~\cite{bnCurves}.
By design, we have $\abs{\G_{1}} = \abs{\G_{2}} = \abs{\G_{T}} = q$.
For efficient implementation, we will need to look at the twist
curve $E'/\F_{p^{2}}$ where

\begin{equation}
    E' : y^{2} = x^{3} + b'.
\end{equation}

\noindent
In this case, $b' = b/\xi$ and the specific choice of
$\xi\in\F_{p^{2}}^{*}$ will be discussed below.
We then define

\begin{align}
    &\psi:E'(\F_{p}^{2})\to E(\F_{p}^{12}) \nonumber\\
    &\parens{x',y'} \mapsto \parens{z^{2}x',z^{3}y'},
\end{align}

\noindent
and we see that $\psi$ is an injective group homomorphism.
Here, we use the definition
$\F_{p^{12}} \equiv \F_{p^{2}}\brackets{X}/\parens{X^{6}-\xi}$,
where $\xi\in\F_{p^{2}}$ is a nonsquare noncube
and $z\in\F_{p^{12}}$ is one of the roots of $X^{6}-\xi$.
That $\xi\in\F_{p^{2}}^{*}$ exists follows from
the fact $p = 1\mod 6$; for more information
see~\cite[Lemma 1]{bnCurves}.
It is this homomorphism $\psi$  which allows for efficient
computation, because this allows most of our arithmetic operations
to occur in $\F_{p}$ and $\F_{p^{2}}$;
in particular, signatures are elements of $E(\F_{p})$
and we represent public keys as elements of $E'(\F_{p^{2}})$.
The only time arithmetic in $\F_{p^{12}}$ is required is when
we compute the Optimal Ate pairing.

We now discuss the specific parameters for the twist curve $E'$:

\begin{align}
    \xi &= i + 9 \nonumber\\
    h_{2,x,i} &= \texttt{198E9393 920D483A 7260BFB7 31FB5D25 F1AA4933 35A9E712 97E485B7 AEF312C2} \nonumber\\
    h_{2,x} &= \texttt{1800DEEF 121F1E76 426A0066 5E5C4479 674322D4 F75EDADD 46DEBD5C D992F6ED} \nonumber\\
    h_{2,y,i} &= \texttt{090689D0 585FF075 EC9E99AD 690C3395 BC4B3133 70B38EF3 55ACDADC D122975B} \nonumber\\
    h_{2,y} &= \texttt{12C85EA5 DB8C6DEB 4AAB7180 8DCB408F E3D1E769 0C43D37B 4CE6CC01 66FA7DAA} \nonumber\\
    h_{2} &= \parens{h_{2,x,i}i + h_{2,x}, h_{2,y,i}i + h_{2,h}}.
\end{align}

\noindent
The values of $\xi$ and $h_{2}$ are from the \texttt{bn256}
implementation used in Ethereum.
We specify the $i$ component before the non-$i$ component
following the library convention.

For BN curves like the one we are using, there is no known
efficiently computable isomorphism
$\vphi:\G_{2}\to\G_{1}$~\cite[Chap 2.2.7]{rass2013cryptography},
which makes our setting Type 3~\cite{EfficienySecurity2010}.

As noted in the Cloudflare \texttt{bn256} source code repository,
the original implementation had approximately 128 bits of security,
but this has been reduced due to recent work~\cite{exTNFS}.
The exact security level has been discussed
here~\cite{MSSsecurity,BDkeysizes}, with~\cite{BDkeysizes} listing the
security level at 100 bits.
While this is below the desired 128-bit level, it is believed
to be currently out of reach.
Even so, with this in mind we take precautions by enforcing
regular key rotation in order to ensure the security of the system.
The decrease in security is partially mitigated by the fact that
validators use the Ethereum private key to sign messages
throughout the consensus algorithm.
