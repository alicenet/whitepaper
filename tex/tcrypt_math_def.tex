\subsection{Mathematical Background and Cryptographic Definitions}
\label{ssec:math_def}

We let $\G_{1}$, $\G_{2}$, and $\G_{T}$ be cyclic groups of order $q$.
Let $g_{1}, h_{1}\in\G_{1}$ and $h_{2}\in\G_{2}$ be generators and
require that the discrete logarithm $\dlog_{g_{1}}h_{1}$ is unknown.
The groups we use were described in Sec.~\ref{ssec:pk_curve_specifics}.
We let $e:\G_{1}\times\G_{2}\to\G_{T}$ be an efficiently computable
nondegenerate bilinear pairing.
In our case, signatures will be elements of $\G_{1}$ while
public keys will be elements of $\G_{2}$.
See Table~\ref{tab:helper_funcs} for additional functions
that will be used in our algorithms.

\input{tables/helper_funcs.tex}

We are building a sidechain on top of Ethereum.
All of the validators will be required to have an Ethereum
public key.
The DKG algorithm requires us to index the participants
from $1$ to $n$.
To do so, we order the participants with respect to their
order of registration.
Our algorithm will need an open broadcast channel;
this will take place via smart contracts on the Ethereum
network.

At times we will want to ensure

\begin{equation}
    e\parens{h_{1}^{\alpha},h_{2}} \overset{?}{=}
    e\parens{h_{1},h_{2}^{\beta}}.
\end{equation}

\noindent
Due to how \textsc{PairingCheck} is defined, we
will need to check the equivalent

\begin{equation}
    e\parens{h_{1}^{\alpha},h_{2}^{-1}}\cdot
        e\parens{h_{1},h_{2}^{\beta}}
        \overset{?}{=} 1.
\end{equation}

\noindent
If $h = \parens{x,y}$, then $h^{-1} = \parens{x,-y}$;
this holds for elliptic curves in Weierstrass form.
For ease of notation, we set $\bar{h} = h^{-1}$.
It may be convenient to store both $h_{2}$ and $\bar{h}_{2}$.

