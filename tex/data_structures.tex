\section{Data Structures}

\subsection{Compact Sparse Merkle Tries}

In order to allow for the desired set of diverse verifiable data
structures that are necessary in a blockchain system, we elected to
select a general purpose tool that could fill most roles in our system
without unacceptably sacrificing performance.
Although more optimal structures may exist for some applications,
forcing developers to build many complex forms of verifiable data
structures complicates initial engineering and future ports to other languages.
Thus, a single structure was selected that could be used in all current cases.
This solution is based on a modified form of the open source Aergo State Trie.
This is a Compact Sparse Merkle Trie that allows for compact proofs of
inclusion/exclusion, stores a leaf at the optimal sub height, and
operates as a base sixteen trie while actually being a true binary trie
implementation.
This polymorphic nature is derived from the manner in which the trie is built.
The length of a generalized encoded proof of inclusion/exclusion
requires $O(\log n)$ merkle proof keys, with an additional overhead of 64
bytes to communicate the key and value of the proof.
The proof also requires $O(\log n)$ bits to communicate the positioning of
the merkle proof keys in the trie.
Thus, the total size is effectively $O(\log n)$.

The trie itself is a binary trie where all nodes are stored in height
four sub tries.
Each sub-trie may be loaded as a batched operation that accommodates a
compact encoding through the use of bit fields that are similar to
those used in the proofs.
Thus, for a given height four sub trie, only nonempty nodes need be
stored, and the bit field may be parsed to determine the positions
these nodes should occupy in the actual trie.
The process of updating the trie may be performed in a concurrent
manner by applying all leaf transformations first and then concurrently
calculating the hashes with a blocking operation on each height four
sub-trie at the leaf levels.
Thus, each sub-sub-trie may be calculated before the root of the
sub-trie itself is calculated, and this concurrency may be arbitrarily
extended across the trie.

\subsection{State Trie}

The State Trie is a CSMT that includes two types of objects and has the
root value of the trie updated and stored in every block.
The first object is the hash of a UTXO stored in the location of its UTXOID.
The second type of object is the hash of a deposit nonce that is stored
in the location that is equal to the deposit nonce.
The first objects, the UTXO type objects, allow a compact proof of
inclusion or exclusion to be formed for the UTXO at a given block height.
This is useful for proofs of datastores, proofs to light clients, and
allows fast syncing of the chain from a known good block, such as a snapshot.
When a UTXO is created, it is added to the trie.
When the UTXO is consumed, it is removed from the trie.
Deposits work in an inverse manner.
When a deposit is spent, it is added to the trie and shall remain there
in perpetuity.
This addition tracks the spending of deposits.

\subsection{BlockHash Trie}

The BlockHash Trie is a CSMT that stores the block hash of each block
mined at a leaf value where the key is the zero padded block number.
This value is included in every block, where the value in each block
references the root of the trie after inclusion of the previous block.
The structure is maintained in order to allow light clients to query
blocks on a singular basis with proof of inclusion for each queried block.
The proofs themselves are intended to start from a snapshot block and
allow previous blocks from the snapshot to be proven as valid blocks
from the entire chain.
This system also has the benefit of allowing proofs of sidechain state
to be formed much more readily in the Ethereum blockchain due to the
ability to easily prove a block header as having a specific block hash
as a fixed complexity procedure.
Alternative systems must either track every block hash in the Ethereum
blockchain, or large batches of contiguous blocks must be injected in
order to prove state at an arbitrary block.
We overcome these limitations through the proofs of inclusion and
exclusion offered by the CSMT.
